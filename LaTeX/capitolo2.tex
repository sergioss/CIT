\chapter{Seconda fase}
\label{seconda fase} % So I can \ref{altrings} later.

\section{Accarezzando il gatto}
\label{accarezzando_il_gatto}

Conducevi una vita tranquilla, orari di lavoro forzatamente normali, non più estremi come prima.

I tuoi avevano allestito una stanza per te dopo che la villa era stata divisa in appartamenti. Era un vecchio garage degli attrezzi, rimesso a nuovo e trasformato in camera da letto.

Un giorno, tranquillo nella tua camera, accarezzavi il gatto, Lalla, che ti faceva le fusa in grembo e guardavi la TV. Su RAI 2 davano uno speciale su Bush all'interno del programma \textit{La Storia Siamo Noi}.

Lo speciale fu veramente critico con Bush e parlava del suo alcolismo, del suo legame con la Bibbia e di strane cose, come il fatto che lui sentiva spesso Dio che gli parlava, che gli diceva cosa e come fare le cose.

Prima della fine dello speciale, iniziasti a sentire qualcosa permeare il tuo corpo dall'alto, un'energia che discendeva su di te.

Per descrivere la cosa che ti successe con gli scarsi elementi a disposizione, allora fu come se una super-intelligenza muovesse tutte le tue sinapsi, connessione per connessione, entrando in simbiosi con te.

Tutto il tuo cervello e, di riflesso, il tuo corpo, erano colmi di quest'energia, che dava una sensazione di tepore diffuso. Smettesti di accarezzare il gatto, sentendolo rozzo sotto le tue mani, come se fosse fatto di una materia grossolana, ordini di grandezza più grossolana di quello che ti era calato addosso, come se fosse fatto di blocchi grossi come il Lego.

Fu come se la cosa dentro di te ti spiegasse la differenza di risoluzione tra il corpo del gatto, che era fatto di materia, e la cosa che invece avevi dentro.

Restasti in uno stato di estasi per un minuto o forse due.

Con quella energia che muoveva intimamente la tua anima, provasti la fusione di due coscienze, una di una risoluzione ordini di grandezza più fine dell'altra, in un unico corpo.

Guardasti l'ora nel cellulare. Il software del telefono Nokia si era bloccato, cosa che non era mai successo prima.

\section{Il negativo rimosso}
\label{il_negativo_rimosso}

Entrare in contatto con quell'energia fu la cosa più toccante della tua vita.

Ebbe anche degli effetti collaterali su di te.

Ti sentivi un altro ed eri molto più concentrato. Altri effetti erano: lucidità, calma, un'infinita pace interiore e l'amore per tutto e per tutti. Ti alzavi presto la mattina e guidavi come un vecchietto, senza alcuna fretta e senza mai imprecare contro nessuno.

Il tuo metabolismo si alterò, eri colmo di quell'energia e il suo calore diffuso ti portava a fare spesso la doccia. Ti sentivi pieno di un'energia magica come quello che avviene alle madri che diventano belle nel periodo della gravidanza, pensavi che non avresti mai avuto più nessun malore e sentivi una sensazione di benessere in cui ogni tua cellula di tutto il tuo corpo splendeva e scoppiavi di salute. Questo durò per più di una settimana, poi pian piano si affievolì, fino a scomparire e per molti mesi non sentisti più alcuna pulsione sessuale.

Restasti puro, come se quella energia si fosse portata via il tuo lato negativo, che non funzionava ormai neanche per autodifesa alla guida. Nessun pensiero malvagio su nessuno poteva più avere origine nella tua mente, come se i pensieri stessi arrivassero e si perdessero, come gocce nel mare, ma senza perturbarti, senza cambiare il tuo pensiero.

Questo durò per due anni, e ogni volta che ci pensi ti rammarichi che questo non sia facilmente replicabile.

È veramente difficile da spiegare. Probabilmente, per sopravvivere nella terra, abbiamo avuto il regalo della mente polarizzata e del nostro lato negativo, che è possibile disabilitare a prezzo di notevoli sacrifici e disciplina. In me invece, successe così come per magia, come due opposte che si uniscono e dalle quali nasce un livello più alto di coscienza, dove il lato negativo è quasi totalmente soppresso.

\section{Maurizio}
\label{maurizio}

Di quel periodo ricordi che tuo fratello ti fece un importante discorso che chiarì in parte la cosa:

“Sergio, la tua mente, durante le ascensioni, è in uno stato particolare: hai tutto integrato. In questi giorni, se non ti dispiace, vorrei che mi facessi compagnia per spiegarti delle cose.”

Così andasti con lui. Fece tappa alla sua abitazione in città, parlandoti del fatto che tanta gente particolare vive di notte, che non sarebbe stato male per te se avessi fatto così, che avresti incontrato un sacco di gente interessante. Poi, guardaste insieme una saracinesca accanto al portone. C'era scritto: “Lavoriamo solo di notte”. Era un sexy shop, e vi metteste a ridere.

\section{Valanga di sincronicità}
\label{valanga_di_sincronicità}

Gli episodi strane in quel periodo non si fermarono.

Te ne ricordi uno in particolare: avevate finito di pranzare da poco e, ad un certo punto, tutti si addormentarono. È stato molto singolare. Gli altri, più che addormentati, sembravano in \textit{trance} e restasti solo tu sveglio. In TV, su \textit{Discovery Channel}, davano un documentario su un buco nero che divorava le stelle, poi dei fisici che correndo armeggiavano con dei macchinari ronzanti.

Un'altra volta eri sul patio di casa a discutere con tuo fratello, per scherzo, sul fatto che avevi due pacchetti di sigarette, uno blu e uno rosso. Tu gli esclamasti: “Questa è materia e questa antimateria.” Entrasti un attimo in casa e, incredibilmente, in TV vedesti due sfere, una rossa ed una blu. Era sempre il canale \textit{Discovery Channel} ma non c'era volume. Lo alzasti e parlavano di materia e di antimateria. La tua vita era sempre piena di queste strane coincidenze, specialmente durante le ascensioni. E, grazie a Franco, imparasti a conoscerne anche il nome: ‘sincronicità’.

\section{Il primo Vangelo}
\label{vangelo}

Tuo fratello ti regalò un Vangelo con la copertina di plastica blu, una versione particolare pubblicato dai Gideons che gli e` stato regalato e nella copertina c'era scritto il nome di tuo fratello. Lui diceva che era una delle traduzioni più attendibili e che non si trovasse in commercio. Non avevi mai letto un Vangelo.

Iniziasti un mini studio del vangelo accompagnato da ricerche in rete. Acquisisti delle informazioni sullo Spirito Santo e la descrizione religiosa coincideva perfettamente con quello che era avvenuto in te.

Ma leggesti anche che lo Spirito Santo non arriva a caso: c'é sempre un motivo e ci doveva essere un motivo per cui ti arrivò durante lo speciale su Bush.

Quindi ritornasti co la mente al distruttore del mare nel bicchiere.

Ti facesti un milione di domande, di cui nessuna aveva risposta, e sorse anche un dubbio: ``Succederà imminentemente? Devo fare qualcosa? Chi sono io per salvare il mondo da questa cosa? Questo non è possibile. Non ho il potere di farlo. Forse le Oil company che investono molto in ricerca hanno già qualcosa pronta che può creare questa devastazione?''.

Ripensando anche a quello che ti era accaduto due anni prima, ti sentisti investito di una missione e pensasti: ``Forse dovrei scrivere qualcosa a Bush. Ma come fare? Bush non legge la posta della gente.''

\section{Greetings department}
\label{greetings}

Mandasti lo stesso un'email alla WhiteHouse.gov. Studiasti un messaggio che non potesse essere cestinato e così ti rivolgesti alla signora Bush. Il messaggio arrivò a destinazione, e venne smistato al Greetings Department della White House. Nel messaggio dicevi che comprendevi le enormi difficoltà e responsabilità del marito. Aggiungevi una preghiera: “Per favore, non fate diventare il mare qualcos'altro.”

Ti arrivarono dei ringraziamenti, forse automatici.

Forse ti rendevi ridicolo ma secondo te c'erano in ballo miliardi di vite. Tentavi di salvare il mondo da questo errore fatale che poteva essere commesso in futuro.

\section{Prima supposizione}
\label{prima supposizione}

La sera della risposta, tuo fratello ti era accanto in auto, rientrando a casa. Lui era come un padre per te, essendo più grande di 16 anni.

Ti fece delle domande, gli parlasti del nuovo progetto di elettronica e lui ti rispose: “Ah, pensavo avessi trovato una nuova fonte di energia. È questo che servirebbe al mondo.” Parlò anche della corsa all'idrogeno e poi disse: “Nel mare ce n'è tanto di idrogeno.” Poi divenne silenzioso.

Era come se sapesse cosa ti frullava dentro: il bicchiere e la lettera alla moglie di Bush.

Nella notte scambiasti dei messaggi con Martin, dicendogli ogni cosa. Inclusi i tuoi dubbi, chiedendogli: “Martin, chi sono io per avere la missione di salvare il mondo?”

Rispose con una immagine in bianco e nero, una incisione di Gesù Cristo che trasportava la croce a spalla, dicendo: “Un uomo può salvare il mondo. Questo è già successo.”

Molti pensieri si legarono dentro di te. “Martin, sono io il Messia e voi Angeli che avete cura di me?”

Lo dicesti essendone convinto al 10\% e non ti saresti mai aspettato la risposta secca e immediata di Martin:

“Sergio, sono veramente contento che sia successo così.”

Ma anche dopo la risposta di Martin, non ci credesti. La tua mente non riusciva ad accettarlo. Troppo grande ed improbabile: 1 su 10 miliardi la possibilità di essere il Messia.

\section{Il volante}
\label{il_volante}

L'indomani tu e tuo fratello eravate andati a comprare materiale assieme all'elettricista, amico di famiglia.

Molti pensieri si accavallavano veloci nella tua mente mentre guidavi, fino a quando capisti parte del puzzle: Il buco nero forse implode, ma poi dà origine ad un universo quando esplode. In quell'istante una stella diventa qualcos'altro, il Dio del suo universo?

Chi sono io? Perché mi è stato chiesto allora se voglio essere Dio o uomo? Se sarò io il presidente del mondo come discutevano, non c'e dubbio su chi io sia.

Facendo i giusti passaggi, la tua mente accettò di essere il Messia.
 
Il peso del mondo ti crollò addosso, non reggesti il pensiero e iniziasti a piangere. Maurizio e l'elettricista, stranamente, non ti dicevano nulla. Dopo un po' l'elettricista ti domandò: “Sergio, perché piangi? Credi di aver fatto qualcosa di male?”

“OK, andiamo. -- intervenne Maurizio -- Torniamo a casa.”

Arrivati in centro, Maurizio ti disse: “Le cose possono sembrare complicate e irraggiungibili, ma non sempre lo sono. Possono diventare semplici se fatte nel modo giusto. Sergio, è come pilotare questa auto all'inizio è difficile poi diventa semplicissimo. Devi abituartici. Vedi, sei tu che hai il volante adesso.”

Ti sembravano dei discorsi strani mentre iniziavi a capire di essere forse il Messia e per quel poco che ne sapevi ti sentivi schiacciato dal peso di dover guidare il mondo.

Questa sensazione ti sembrò troppo e provasti un'enorme paura. Maurizio, ti urlò: “SERGIO, IO NON SONO TUO FRATELLO.”

Lo guardasti incredulo, poi accostasti velocemente di fianco ad un muro e uscisti in un istante dalla vettura. Dandotela a gambe levate, corresti, corresti, senza fermarti, il più veloce che potevi.

Arrivasti al Palazzo Beneventano. Era lì che tuo padre aveva lavorato una vita. “Un posto sicuro”, pensasti.

Ricevesti dei messaggi e una telefonata al tuo cellulare ma non era tuo fratello.

Eri molto confuso. Non sapevi cosa sarebbe successo.

\section{L'esorcismo}
\label{esorcismo}

Decidesti dopo un po' di andare nella chiesa sulla grande piazza lì vicino. Non correvi più. Ti sentivi fuori dal mondo. Nella chiesa mezza vuota, un prete, con la stola viola, stava celebrando la messa e le parole del prete ti colpirono: parlava della seconda venuta del Signore.

Ad un certo punto sentisti tutto il male del mondo concentrato dietro alle tue spalle, la sensazione del male puro concentrato in un punto proprio dietro di te. La porticina della chiesa si aprì: era una vecchietta.

Ti scagliasti su di lei, tirandola indietro, cercando di farle varcare la soglia esterna della chiesa, con tutta la forza che avevi. “Il male fuori dalla chiesa!” urlasti, ma nessuno capiva cosa stessi facendo. Otto persone del mercato all'aperto si fiondarono su di te credendoti uno scippatore. Solo una persona ti accarezzò il cuore, una prima volta e poi più volte consecutive. Ancora oggi, non sai chi fu che ti passò una mano dolcemente sul cuore e sul petto, più e più volte, come se sapesse quello che stavi cercando di fare, mentre gli altri ti tiravano dentro da ogni parte del tuo corpo. Tirasti più forte che potevi, ma le persone su di te erano sempre di più, ti accasciasti a terra sfinito.

Quando riprendesti i sensi, c'era un prete vestito di nero sopra la tua faccia, con gli occhi marroni. Stava dicendo delle formule rituali che non capivi e ti osservava aggrottato, poi disse: “Accetti l'aiuto di Nostro Signore Gesù Cristo?” Lo ripeté più volte. “Non ho fatto niente di male.”, dicesti.

Cercasti di rialzarti e colpisti uno che ti porse gli occhiali. Ti sentivi attaccato da tutte le parti. Rompesti gli occhiali in due pezzi e poi ti allontanasti. Un poliziotto ti venne dietro, ti fece voltare e ti colpì con un pugno da karateka sul muso. Iniziasti a sanguinare copiosamente. Sentivi, mentre la polizia ti portava via ammanettato, che dicevano che il portafogli della vecchietta aggredita era stato rubato. Ti avrebbero incarcerato, pensasti.

Eri nella volante della polizia. Lo schienale era durissimo. Non c'era imbottitura. Era tutto di plastica, vetroresina, o quello che era.

Sentivi una pace infinita dentro e nessuna paura per quello che sarebbe successo dopo.

Un infermiere con un crocefisso come orecchino ti tamponava il muso grondante di sangue.

Guido, tuo amico da vent'anni, ti venne a riprendere al commissariato. La signora non sporse denuncia, si procurò solo un taglietto, inciampando sulla porta. Così ti lasciarono andare.

Guido non disse una parola lungo il tragitto.

\section{Il Rifiuto}
\label{il_rifiuto}

Ti rendesti conto di quello che avevi fatto: prendersela con una vecchietta solo per quella sensazione che avevi avuto.

Decidesti che non stavi bene ed era meglio evitare di uscire di casa per un po'. Parlasti con tuo fratello e gli dicesti di volere una opinione diversa da quella di Barbara, fu così che da li a poco, ti accompagnò da uno psichiatra.

Dopo che raccontasti un sunto delle cose che ti erano accadute dal 2002 in poi, del nuovo fatto dell'energia calata su di te e di come ti fossi sentito dopo, lo psichiatra ti pose qualche domanda:

“A che distanza, in termini temporali, dal primo episodio del Buco Nero è arrivata questa energia che le è caduta addosso?”

Rispondesti: “Circa due anni.”

“Quanto è durato l'effetto di questa energia?”

Rispondesti: “Circa due settimane.”

“Hmmm, Credo che lei sia affetto da Ciclotimia o disturbo bipolare. Chi soffre di questa condizione tende a presentare fasi depressive seguite da fasi maniacali e queste possono insorgere anche istantaneamente. Generalmente le fasi depressive tendono a durare maggiormente rispetto a quelle maniacali. Di solito, le prime durano da qualche settimana a qualche mese mentre le seconde da una a due settimane. Lei è stato depresso?”

Rispondesti: “Si, da morire dopo l'evento del 2002.”

“Bene. Credo che non ci siano dubbi. Lei è bipolare. Molta gente lo è, e conduce una vita quasi normale, a parte gli episodi di crisi che possono verificarsi o non verificarsi se si segue una terapia farmacologica adeguata. Dobbiamo cercare di evitare questi episodi perché le compromettono la mente. A tal fine le consiglio di assumere giornalmente uno stabilizzatore dell'umore. Il Depakote dovrebbe andare bene. Commercialmente si chiama Depakin. Una compressa da 200mg due volte al giorno.”

Tuo fratello lo ringraziò, così facesti anche tu e ve ne andaste.

“È una liberazione. -- dicesti, entrando in macchina -- Ora so il perché di tutte le cose che mi sono accadute. Spero solo che tutti i cazzo di orologi che vedo non siano più con le cifre simmetriche.”

Lui rispose: “Siamo venuti qui solo per avere un'opinione diversa da quella di Barbara. Lo hai chiesto tu.”

E fu così che ti togliesti dalla mente quelle strane idee che ti erano venute. Era semplicemente impossibile; eri tutto tranne che megalomane. Essere Cristo! Come tanti altri, avevi sempre avuto enorme rispetto per Lui. Forse lo avevi creduto per come ti eri sentito dentro? Anzi, eri sicuro che, se fosse durato a lungo quel modo di essere e di sentirti, non avresti avuto dubbi.

Quella vecchietta cercava solo un po' di tranquillità e purificazione del suo animo in chiesa. Decidesti che non eri degno, che era tutta un'enorme coincidenza, che non potevi essere tu, Dio. Archiviasti tutto e te lo buttasti dietro le spalle.
