\chapter{Terza fase}
\label{terza_fase} % So I can \ref{altrings} later.

\section{I Roditori di Marte}
\label{rodotori_di_marte}

Ti dedicasti al lavoro, gestivi anche un piccolo negozio di informatica con assistenza tecnica. Quando iniziasti, ci fu un vero boom con le modifiche delle consolle PlayStation. Lo facevi anche come lavoro sociale e su ogni modifica mettevi un bollino con la scritta “L'informazione deve essere libera”.

Una volta pensasti che, probabilmente, quello che è il gioco per i bambini di oggi sarà lavoro e vita quotidiana negli anni a venire, forse nelle città virtuali dell'internet futuro.

Era maggio 2007 quando decidesti di ristrutturare il negozio. Il tuo amico ciabattino di 93 anni spirò pochi mesi prima.

Eravate amici nonostante la differenza di età. Spesso, lui veniva a cambiare le banconote e aveva la fissazione di tenersi i tagli piccoli, ma ricordi che lo facevi sempre con piacere. Più volte te lo aveva detto: “Sergio, quando non ci sarò più vorrei che tu prendessi la mia bottega.”

Così la prendesti in affitto, e buttasti giù una parete del negozio per unificare le botteghe.

Per coprire il muro della facciata pesantemente rovinato, facesti realizzare una scultura in polistirolo, tela e colla sabbiata di un uomo vitruviano, con delle modifiche. Il quadrato dell'immagine divenne un ottagono inscritto nel cerchio e l'uomo aveva alla sua destra una forma d'onda continua, e alla sua sinistra uno stream di bit.

Eri dentro al negozio. L'elettricista, Silvio, lavorava all'impianto elettrico con un giravite col numero 4 impresso.

Ti mettesti in una sedia di fortuna a navigare su internet, seguendo dei link sulle novità nel settore dell'energia e ti imbattesti su delle ricerche finanziate dai governi e dalle Major del petrolio su delle alghe modificate che producevano idrogeno.

Sempre tenendo presente il bicchiere distruttivo, pensasti che le alghe che producevano idrogeno dall'acqua di mare, potevano solo indirettamente aumentare l'effetto serra se lasciate libere di riprodursi in natura, ma non distruggere il mare. La terra avrebbe avuto grossi scompensi, ma un sistema così complesso doveva pur avere dei sistemi di protezione.

Pensasti in un lampo che forse la cosa dell'intruglio nel bicchiere non doveva accadere ma era già accaduta, che non era proprio successa sulla Terra, che noi non eravamo così avanzati tecnologicamente.

Poi guardasti il tuo notebook con un raytrace della sonda Spirit che era alla ricerca di acqua su Marte. La avevi come wallpaper da qualche mese sul desktop Gnome e ti venne l'idea che forse era già successo, ma su Marte.

Forse la soluzione a questo tarlo, che avevi da anni, ce l'avevi davanti agli occhi.

Mentre pensavi questo e guardavi lo schermo del notebook, suonò il telefono e un'altra sincronicità accadde.

Era un cliente, che ti chiese: “Avete il gioco per PlayStation \textit{Rodents from Mars}?”. Non avevi mai sentito prima quel gioco.

Cercasti sul web di quel gioco, che non avevi mai sentito nominare prima, incredulo di quella coincidenza. Effettivamente il gioco esisteva e si chiamava \textit{Biker Mice from Mars}. “Sono iniziate di nuovo le sincronicità.” pensasti.

\section{La comprensione parziale}
\label{comprensione_parziale}

L'ultima volta eri finito al commissariato e stavi per fare del male ad una vecchietta, così decidesti di startene a casa. “Lì -- pensasti -- non può succedere niente di grave.”

Ma succedevano delle cose strane, anche lì.

Durante il periodo delle ascensioni, in cui stavi forzatamente in casa per non incappare in rischi inutili, facevi delle chiacchierate con tua sorella, tuo fratello e tuo padre, anche se lui parlava pochissimo, quasi niente per il suo inizio di Alzheimer.

Un giorni eri seduto con tuo padre sul davanzale della villa a prendere un po' di sole dopo pranzo. A un certo punto, ti chiese di guardare un punto nel cielo. “Guarda lì," disse, indicandolo con un dito.

C'era qualocosa di strano: le nuvole in quel punto erano immobili.

Poi lui fece con la mano un gesto, come per spostarle e, incredibilmente, si spostarono e cambiarono configurazione.  Restasti a bocca aperta, incredulo.

Un pomeriggio eri con tua sorella, seduto all'ingresso della spiaggia su una scivola di cemento armato tutta fracassata dalla forza del mare, realizzata decenni prima con del cemento scadente. Ad un tratto le chiedesti: “Titty, pensi ci sia stata vita su Marte un tempo?” Lei ti rispose: “Cosa era Marte se non la casa dell'uomo prima della Terra?”

Restasti in silenzio. Ogni volta che ti accadevano queste cose ti sentivi scosso nel profondo e restavi di sasso, anche un po' incredulo. Davanti al pozzo dell'Amenano avevi però deciso che questa vita che facevi era reale; era la tua scommessa sulla vita. “Non sto sognando” ti dicevi.

Ti convincesti che era solo la tua comprensione attuale del mondo che ti faceva sbandare. Perché mai tua sorella doveva dirti una cosa del genere? Chi parlava per mezzo di lei?

Ricordi anche che, dopo mesi, quando volesti accertarti di quello che disse, negò di averlo detto. E subito dopo lo ammise. “Strano” pensasti.

Si facevano discorsi a quattro: tu, tuo padre, tua sorella e tuo fratello. Più che altro, tu ascoltavi cosa si dicevano. Discutevate sull'uomo, sul fatto che delle entità intelligenti ci collezionerebbero tutti, che siamo ognuno diverso dall'altro, che anche se la vita finisce in un mondo, l'uomo non può mai estinguersi perché viene preservata la specie, che non si dovrebbe giocare con la biotecnologia, ma andrebbe usata solo a scopo medico.

E altri frammenti: sul tempo, sul fatto che aprire un wormhole non crea danni, sullo spazio che si auto ripara, sui viaggi temporali in cui andare indietro nel tempo è tassativamente vietato, a meno che non si sia solo degli osservatori senza corpo e senza influenza, sui miracoli, su come si cura una persona portando indietro nel tempo la parte malata; che la materia conserva la memoria degli stati precedenti.

Anche dalla vicina, Soraya, tuo fratello una volta ti chiamò per parlarti con lei di quel caso del bambino che ricordava con esattezza i dettagli di un'altra vita. Loro ti spiegarono che può succedere che è un errore, come lo è la depressione che avviene quando corpo e anima divergono, come se si volessero scusare per queste cose.%\looseness=-1 %Avoid widow "cose."

\section{La pianta della Coca-Cola}
\label{pianta_della_coca_cola}

Ormai, dopo quello che avevi passato, niente ti sembrava più strano. Parlando di biotecnologia, una volta, tuo fratello ti fece un regalo, una bottiglia con una pianta dentro, tu gli chiedesti: ``Cos'è?'' e ti rispose: “Come, non lo vedi? È la pianta della Coca-Cola.”.

In quel momento ricordasti il giorno in cui ti venne a trovare Ivan in negozio e ti disse: “Sergio, il mondo va a rotoli. Sai che ci sono un mare di porcherie fatte dalle aziende farmaceutiche? Una di queste è la creazione di virus a tavolino. Creano i virus e li diffondono in vari modi per poi vendere i vaccini”.

Restasti schifato e amareggiato, poi Ivan disse ridendo: “Sai come si può risolvere la cosa? Dobbiamo diventare tutti gay”. Non capisti granché. Evidentemente non si riferiva alle donne, ma forse agli opposti in noi e al nostro lato malvagio, da sempre incombente.

\section{Modello della Superficie}
\label{modello_di_superficie}

Finalmente, la notte che è al centro di questo racconto arrivò.

Quella notte lavoravi al web di Sim.One. Lasciasti il tuo notebook acceso sul letto per andare fuori a fumare una sigaretta. Notasti che, chiudendo gli occhi, continuavi a vedere le cose. Vedevi le tue mani muoversi ma delle tue dita una sagoma soltanto. Ti appariva così tutto ciò che vive, comprese le piante del vialetto. Facendo ulteriori prove per una buona decina di minuti, tutto coincideva.

Rientrasti dentro casa e decidesti che era troppo reale per essere un'allucinazione. Avevi fatto delle prove, avevi contato con le dita, avevi camminato a zonzo a occhi chiusi e ti eri avvicinato alle piante. Ad occhi chiusi tutto era sagomato in quel grigio più chiaro del buio -- era incredibile -- tutto era al posto giusto, incluse le tue dita quando le muovevi.

Ti stendesti a letto, accendesti il PC e decidesti di ascoltare un po' di musica: avevi scaricato da poco qualche pezzo dei Deep Purple dopo aver visto il concerto a Catania. Avevi solo un pezzo però nel notebook: \textit{Child In Time}.

Lo facesti partire.

La musica partì e anche il Mediaplayer.

Guardandolo iniziasti a spaziare con la mente, guardando il feedback dell'audio come se avesse un significato interno, come se fosse stato un tutoriale sull'universo, come se ti spiegasse pian piano tutte le cose. Guardavi estasiato gli effetti della grafica del Mediaplayer di Gnome e pensasti: ``Wow! Tutto questo dentro un pezzo dei Deep Purple!''

La musica si ripeteva ciclicamente, non ricordi quanto tempo passò ma la tua mente accelerò al massimo e si sbloccò del tutto.

Prendesti qualche appunto in un file testuale:

\begin{quote}
{\ttfamily\small
Le forze:

viola, bianco, giallo, nero

La composizione:

positivo, negativo, informazione, freccia-del-tempo

Leggi anche la mente è tutto.

Ho staccato appena in tempo \ldots
}
\end{quote}

Al culmine scrivesti queste e altre poche righe in un'altro file:

\begin{quote}
{\ttfamily\small

La Mente è tutto.

Le stelle agiscono da neuroni.

I demoni risiedono nelle lune.

Io ero il Sole una volta \ldots
}
\end{quote}

Il tuo cuore accelerò i battiti in maniera furiosa. Riuscisti a spegnere il PC, riuscendo a vedere a mala pena il tasto di spegnimento tra le lacrime. Il PC si spense, ma non quello che ti stava accadendo.

Eri al buio, steso sul letto, cercavi di frenare.

Milioni di puntini si accesero contemporaneamente nel tuo torace, e una volta accesi, era come se pregassero tutti insieme\ldots Stelle?

Delle strane forme come i disegni di melanzane tagliate a metà, che fluttuavano e che si distorcevano, rimanevano impresse nella tua retina o nello schema usato nella tua mente per essa, come quando vedesti le tue mani in una semi-visione di un grigio poco più chiaro del nero.

Le forme somigliavano a quello che vedesti molti anni dopo, in un modello di sintesi della superficie del sole. Incredibilmente, era proprio quello che vedesti, solo che allora non capivi ancora cosa fosse.

Comunque, provavi a non perdere quelle forme in movimento, sapendo che era importante. Pensasti: ``Proverò a disegnarle.''.

Poi ti alzasti dal letto e l'immagine grigia a occhi chiusi cambiò, scorrevi velocissimo in un wormhole. ``Incredibile. -- pensasti di nuovo -- Se lo vedesse chi li ha teorizzati!'' Era come quello dei film di fantascienza, piu` o meno.

Volevi che finisse, eri arrivato al limite, forse eri anche andato troppo oltre. Cercasti aiuto da tua sorella al piano di sopra. Eri sconvolto, non avevi quasi più voce, era come se avessi partorito migliaia di stelle.

Lei aprì la porta e ti chiese: “Sergio, cos'è successo?”

Le dicesti che ti eri sentito male ma che stava passando, che volevi un po' di compagnia. Lei allora ti prese la mano e ti stette accanto qualche ora.

Quella notte sapevi che qualcosa era cambiato, che avevi raggiunto un traguardo. La mattina dopo, pensavi agli antichi egiziani, che forse c'era del vero nei loro culti solari. Ti sentivi accidentalmente un faraone, anche se senza regno, né scarpe.  Capisiti che ti era stato affidato un messaggio, che questa volta era stato molto più completo dei precedenti, ma che ne avevi compreso un'infinitesimale frazione di esso.  Per rispetto, o forse per apatia, sapevi di esserti fermato al momento giusto o forse poco più in là, prima che il tuo cuore cedesse.

Ogni volta che guardavi in cielo, sentivi un profondo legame con il sole e ti sentivi anche lì, non solo qui.  Questa forte sensazione di legame con il sole durò qualche giorno, poi svanì pian piano.

I tuoi ti consigliarono di restare a casa. Facevi cose strane, dei giri di corsa circolari attorno alla casa. Camminare in circolo ti faceva sentire bene e andavi sempre in un senso ma non ricordi se era orario o antiorario.  Scrivevi dei cartelli e li posavi a terra: dei messaggi al Sole e la cosa ti fa ancora sorridere\ldots

\section{Il Vecchietto alla Posta}
\label{il_vecchietto}

Passò circa un mese. Era la fine di giugno e le cose strane erano passate da un pezzo. Anche questa volta la sensazione di purificazione era forte.

Iniziasti a lavorare nuovamente.

Fu così che una mattina andasti con tua sorella e tuo cognato alla posta della zona industriale, vicino agli stabilimenti della ST Microelectronics per spedire dei pacchetti con raccomandata estera.

Loro entrarono, tu restasti fuori per prendere un po' d'aria, eri appoggiato alla ringhiera usurata e senza vernice, davanti alla porta e guardavi verso la strada. Alla tua sinistra c'erano delle persone e ad un certo punto passò davanti alla ringhiera, che poggiava su un pavimento di cemento sopraelevato, un vecchietto che ti chiese un'informazione:

“Mi scusi, mi sa dire che giorno è oggi?” Di solito non rammenti la data e gli rispondesti: “Mi dispiace, non lo so”. Ad un certo punto lui disse: “Oggi è il 22 di giugno. Tra tre giorni è Natale. Grazie comunque” e se ne andò alzando il cappello per salutare.

Sei nato il 25 giugno del 74, 6 mesi dopo il Natale.

Non lo dimenticherai mai: restasti di pietra. Poi ti girasti a sinistra e fortemente emozionato cercasti appoggio in qualcuno, dicendo a voce alta: “Avete sentito che ha detto il vecchietto?” Ma la gente non ti diede retta.

Questa ``lettera'' verbale ti arrivò da un vecchietto che non avevi mai visto, 3 giorni prima che compissi 33 anni.

\section{I DVD in testa}
\label{dvd_in_testa}

Dopo un mese o poco più dall'episodio del vecchietto, decidesti che stavi per scoppiare, che dovevi raccontare questa storia a qualcuno.

Ti incontrasti con Alessandra, una ragazza che frequenti tutt'ora, e le dicesti: “Credo di essere il Messia” ed il perché, punto per punto. Gli raccontasti un po' della storia, lei ti disse che Matrix non ti avrebbe aiutato a capire, che avresti dovuto vedere il film \textit{Il Tredicesimo Piano} per capirci qualcosa in più.

Quel film mostra infatti una realtà a più livelli in cui i livelli più alti creano un sotto mondo, nel sotto mondo avviene di nuovo la stessa cosa e le entità dei mondi più alti controllano quelle dei mondi più bassi.

Le chiedesti allora: “Siamo figli di altre civiltà?” Lei ti rispose: “La botta che hai preso oggi non ti è bastata?”, che ti erano caduti infatti dei cartoni di DVD vergini in testa mentre scaricavi un pallet per farlo entrare nel negozio.
