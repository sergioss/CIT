\chapter{Il mare nel bicchiere}
\label{il_mare_nel_bicchiere} % So I can \ref{altrings} later.

\section{Peter}
\label{defs}

Alla fine del 2002, il tuo amico Peter era da poco rientrato in Taiwan dal suo viaggio a Catania. In quel periodo scambiavi delle email con Peter il cui contenuto era mirato a descrivere la condizione generale della civiltà moderna, descrivendo con essa anche un'immagine -- un asino che inseguiva una carota su una canna da pesca -- quando l'email che probabilmente cambiò la tua vita venne fuori\ldots

\begin{quote}
{\ttfamily\small
Peter,

ti dico una cosa sulla stupidità umana:

Sant'Agostino diceva: "Non puoi mettere il mare dentro il bicchiere". Posso cambiare questa cosa in "la nostra stupidita' e' cosi' grande che possiamo mettere nel bicchiere qualcosa per distruggere il mare intero" \ldots
}
\end{quote}

Era uscita di getto, l'email a Peter, e non ne avevi compreso appieno l'importanza. L'idea del mare devastato da un intruglio presente in un bicchiere non ti lasciava in pace. Come appassionato lettore da diversi anni di news sulla nanotecnologia, capisti l'importanza di quello che avevi scritto, anche se tardivamente. La corsa all'idrogeno avrebbe potuto spingere la ricerca nanotecnologica in direzioni pericolose per l'ambiente, creando qualcosa che, replicandosi, potesse innescare un processo di separazione di ossigeno e idrogeno in tutto il mare.

\section{Trapasso}
\label{trapasso}

Trascorsa una settimana da quando era andato via Peter, non avevi più pensato all'intruglio nel bicchiere. Dormivi poco, fumavi molto e eri molto concentrato sul lavoro.

Cristina ti venne a trovare nell'ex ufficio del webmaster dove stavi facendo un po' di lavoro su una workstation Linux. Ti lasciò un caffé e la ringraziasti.

Era veramente una brava ragazza, Cristina. Lavorava per te part time, sapevi che aveva qualche problema economico, eri contento di averla assunta, anche se non sapevi se potevi mantenerle un impiego per molto. Le cose non andavano granché bene.

Dopo il caffé decisi di fare un break e andasti a leggere qualcosa di tecnico su slashdot.org. Era un blog che leggevi spesso e ti eri anche iscritto da poco con il nick \textit{orangehead.}

Leggendo tra le righe dei commenti, iniziasti a dubitare se stessi dormendo o se fossi sveglio: stavano parlando di te! Era proprio un dibattito su quello che avevi fatto e su quello che facevi. Qualcuno dopo ti chiamò `world president'\ldots tu, presidente del mondo? Ti veniva anche da ridere per quest'assurdità, anche se ti sembrava stranissimo che queste persone sconosciute parlassero di te.

Ti alzasti in piedi, ed ecco che succese qualcosa di inevitabile: un vortice di immane potenza iniziò a girare nella tua testa, restasti con le mani poggiate saldamente al tavolo e non ti potevi muovere. Eri tutto il vortice, sentivi tutta la materia di cui era fatto, la sua forza era veramente immane. Il vortice si ridusse e si aumentò di velocità fino a risucchiare tutto e a diventare un puntino e sentivi tutto te stesso, con montagne e montagne di informazione, risucchiato in un piccolo puntino.

Poi sentisti una domanda in un tempo al di là del tempo, senza nessuna flessione, in un modo di parlare che non avevi mai sentito, senza nessun accento: “Vuoi essere uomo o vuoi essere Dio?”

Rispondesti sinceramente: “Uomo”.

Ti sentisti urlare ed era come se fossi tornato indietro e stessi cercando di contrastare la forza del vortice con un urlo che ti veniva da dentro con una forza enorme che si irradiava da te.

Tutto fu fatto, non sapevi cosa, ma eri sopravvissuto.

\section{Bulldozer}
\label{bulldozer} 

Non potevi parlare, né camminare bene. Scendesti le scale barcollando, farfugliasti qualcosa alle ragazze, cascasti a terra e ti sentisti afferrare.

Ti risvegliasti nella stanza al piano di sopra. Entrò Alma, una delle tue impiegate, che venne a vedere come ti sentivi. Le dicesti che stavi bene, anche se ti sentivi la testa rasa al suolo da un bulldozer. Sentivi i rumori tutti accentuati e ovattati, forti ma senza timbrica e senza dinamica.

Alma aveva uno strano bagliore negli occhi, ti guardava in modo strano, come se non fosse lei, e chiuse la porta dolcemente.

Ti riaddormentasti\ldots

Tuo fratello Davide, medico, e Elisa, una neurologa sua amica, entrarono nella stanza e ti chiesero cosa fosse successo.

Gli raccontasti del buco nero in testa, e farfugliasti delle cose su una chiave di compressione frattale, su montagne di informazioni che si erano ridotte ad un puntino, sul tempo che si era ripiegato su due mondi che si erano uniti.

Elisa portò un flacone scuro, con la scritta Rispedal, ti chiesero di prenderlo, che ti saresti sentito meglio dopo. Lo mandasti giù.

Isabella, la tua compagna era molto spaventata, piangeva e ti evitava.

Vedevi una strana luce negli occhi di Cristina e di Alma, come dei riflessi metallizzati. Notasti che i loro occhi erano molto, molto profondi, un abisso più del solito.

Passarono alcuni giorni e ti interrogavi su quello che era successo.

Ad un certo punto avessi un'illuminazione e ti venne un'idea per giustificare l'accaduto. Non sapevi da dove, non sapevi da quando, ma qualcosa era cambiato per quello che avresti fatto in futuro. Da appassionato di nanotech credevi di aver fatto qualcosa di irreparabile e di grave proiettata nel futuro e ti sentivi in colpa.

Andasti alla tua postazione in ufficio e stampasti un foglio A4 con sopra scritto: “Scusate per quello che ho fatto”. Lo appendesti in bella evidenza sulla porta di ingresso del negozio all'ingrosso ed uscisti in strada.

Camminavi e cerchavi Colin, un hacker che avevi conosciuto molti anni prima. Sapevi che non era più in Italia ma arrivasti comunque in via Etnea, vicino al palazzo dove una volta abitava, e lì cadesti a terra, esausto.

Eri distrutto, non sapevi cosa ti stesse succedendo, volevi che Colin fosse stato lì. Solo lui, pensavi, poteva capire meglio quello che ti fosse realmente accaduto.

Ritornasti in azienda dove la tua compagna ti chiese delle spiegazioni per il cartello. Le dicesti che ti sentivi distrutto, che non sapevi cosa ti stesse succedendo. Dal tuo ufficio sentisti che lei chiamò qualcuno dicendogli del cartello. La cosa ti dispiaque.

\section{Caos e opposti}
\label{caos_e_opposti}

Uscisti e andasti al chiosco delle bibite lì vicino. Ordinasti un selz e limone e ti accorgesti che l'impiegato del chiosco aveva dei riflessi negli occhi simili a quelli di Cristina e di Alma. L'uomo del chiosco ti chiese “Sergio, come stai?”. Gli dicesti che ti sentivi come se fosti passato sotto un trattore. Lui ti servì la bevanda e ti disse “Sergio, bevi questo e ti sentirai meglio.” La bevesti, incredulo a queste parole di circostanza.

Non ti fece pagare e lo ringraziasti.

Ritornasti in ufficio, accendesti il PC e ti sedesti, stanco e devastato.

Ti rendesti conto che il tuo cervello non funzionava bene, che era un caos.

Non sapevi cosa fare e apristi delle immagini a caso. Ti soffermasti su una in particolare, fatta dal team che lavora alla grafica delle scatole delle licenze di Windows. Era un'immagine sul nanotech e ti faceva un po' di paura. Apristi altre immagini fin quando ne apristi una che fece una strana magia. Mostrava due mani che offrivano due cose opposte: a sinistra un Tux di Linux e a destra il logo di Windows.

Guardasti fisso l'immagine, e ne traesti subito un senso di sollievo. Era come se ristabilisse gli opposti nel caos della tua mente. Oltre ad un profondo sollievo, sentisti partire due spirali gentili che si muovevano ritmicamente nei due emisferi del tuo cervello, una per emisfero, una specie di movimento ritmico a impulsi ciclici. Questa cosa ti donò un benessere enorme che si espanse in tutto il tuo essere e nella tua mente. Ti mettesti allora ad ascoltare un po' di musica, che ti faceva provare delle fortissime emozioni, come se pian piano ti si ricostruisse la mente.

\section{Primo contatto}
\label{primo_contatto}

Elisa, la neurologa, ti consigliò di non lavorare finché non ti saresti sentito meglio, e di non guidare. Così tornasti a casa, avendo passato i primi giorni di recovero nei locali dell'ingrosso dove al primo piano c'erano due stanze da letto.

Passasti due settimane a casa e, da quando tutto era iniziato, parlavi pochissimo. La tua compagna era sempre molto intimorita e servizievole; tu ti sentivi vuoto come se la tua mente fosse passato da un puntino e ricostruita dall'altro lato, mezza sconquassata.

Per proteggere la tua mente, ti avevano dato una cura di Rispedal e lo prendevi due volte al giorno.

Dopo le due settimane di ricovero a casa riprendesti a guidare la tua automobile, una Smart del '98. Quando salisti, l'orologio sul cruscotto segnava le 11:11 e pensasti che uno di quei giorni lo avresti dovuto sistemare.

Arrivato vicino al carcere di piazza Lanza, dove càerano i cinema all'aperto, incontrasti Cristina. Non aveva più il bagliore negli occhi e si stava recando al lavoro, come ogni mattina.

Accostasti, la feci salire e lei ti chiese come stavi.

Le dicesti “Bene”, per formalità.

Poi guardaste il traffico davanti a voi dove, in pochi istanti, si era creato un ingorgo con centinaia di macchine ferme: un blocco totale.

Lei ti guardò per un po' e ti disse: “Dimmi, Sergio, questo ingorgo ti sembra normale?”

Le risposi secco: “No, qui non c'e mai confusione.”

“Posso farti un'altra domanda? Rispondimi sin\-ce\-ra\-mente. Tu, di che religione sei?”

Le risposi che apprezzavi un po' tutte le religioni, ma che non ti sentivi legato a nessuna in particolare.

“E tu?”, le chiesi. Lei rispose sorridendo: “Credo in Cristo.” Poi aggiunse: “OK. Scendo qua\ldots arrivo prima a piedi.”

Appena scese, immediatamente le auto iniziarono a muoversi e l'ingorgo iniziò a fluire.

La tue mente non era al massimo della sua prontezza, ma fu allora che ti accorgesti, per la prima volta, che delle entità spirituali ti stavano intorno e ti parlavano attraverso le persone. Non sapevi chi o cosa fossero, ma sentivi che erano amichevoli. Per circa due giorni, potevi vederli dagli occhi metallizzati, poi la visione scomparve.

Ma questa cosa, la prima volta, non ebbe un bel effetto su di te.

\section{The Matrix}
\label{the_matrix}

Arrivasti in ufficio, ma uscisti subito dopo.

La realtà ti sembrava finta, di carta, un mondo costruito alla Matrix.

Così girasti tutto l'isolato, cercando qualche traccia per uscire fuori dalla realtà finta in cui ti trovavi. Eri certo, oramai, che il mondo che conoscevi era una presa in giro: c'erano troppe coincidenze impossibili e volevi che questi amici dagli occhi profondi ti tirassero fuori da un mondo che credevi finto.

Non era come Matrix, però, anche se era la cosa più plausibile, per te, in quel momento. Ritornasti a casa, poi si fece sera e uscisti nuovamente a piedi, girovagando in cerca di una risposta. Eri confuso e non capivi cosa stesse succedendo. Così, camminando, ti ritrovasti in piazza Duomo.

La piazza era in riallestimento: un cantiere tutto recintato. Scavalcasti il recinto.

Dovevi uscire da questa schifezza finta dove tutto aveva il sapore di patata. Dovevi tornare al mondo vero. Così pensasti di buttarti dentro il pozzo nero del fiume Amenano. Scendesti nella fontana di sotto e, laddove fluiva l'acqua, una voragine si apriva davanti a te e stavi per farlo. Stavi per buttarti. Poi sentisti qualcuno passare parlare di sant'Agata e non ti buttasti. Non ricordi esattamente cosa disse su di lei ma quello che disse non ti fece buttare dentro al pozzo.

Mentre uscivi dalla fontana due lavoratori ti urlarono qualcosa, scappasti scavalcando la recinzione procurandoti delle piccole ferite.

Rientrasti tutto sporco \ldots

Triplicarono la dose di Rispedal ed Elisa parlava di una ricaduta, ma con quella dose massicia non parlavi più e seguivi le persone dappertutto, qualsiasi cosa facessero, come uno zombie che non ha volontà.

\section{Matrimonio}
\label{matrimonio}

Due mesi dopo, a dicembre del 2002, preparavi i documenti per sposarti.

L'impiegato del comune di Gravina, mentre vi accompagnava nel suo ufficio, iniziò un strano discorso infarcito di luoghi comuni, che non ricordi benissimo. A brani, ricordi diceva: “Il mondo è giunto ad una svolta.” “Le cose cambieranno presto.” “Il potere amministrativo sarà affidato alle donne.” “Ci sono cose che succedono ogni 2000 anni.” “Tutto questo è una cosa perfettamente normale.”

In quel momento non capisti granché di quel discorso.

Trascorresti tutto l'inverno in uno stato di forte depressione. Non leggevi più le riviste hi-tech che ti piacevano. Avevi paura ad usare un computer. Iniziasti a fare delle sedute da uno psicoterapeuta.

\section{Gestalt}
\label{gestalt}

Il tuo psicoterapeuta si trovava nella citta vicina, Siracusa, così ti recavi lì due volte a settimana.

Ogni volta che raggiungevi Siracusa, ti rilassavi. Era come se tutto fosse più lento, meno caotico e più gentile.

Lo studio di Franco era tappezzato di titoli, tra cui un master USA sulla psicologia della Gestalt.

Ti faceva sempre parlare delle cose quotidiane e alla fine ti diceva: “Prendi contatto con la natura. Esci e vatti a fare una birra con un amico.”

Quando prendesti confidenza, gli accennasti del buco nero in testa, lui ti rispose in questo modo:

“Sergio, è indubbio che hai fatto un viaggio e sei andato molto in alto, molto più in alto di chiunque altro. Considerati come un supereroe.”

Una volta, mentre ti recavi nel suo studio, ti venne incontro un pulman con una scritta curiosa e il nome Franco scritto a lettere cubitali. Lui si chiama Franco, quindi gli raccontasti di questa e di tante altre coincidenze che ti accadevano in continuazione, incluso il guardare cartelli pubblicitari proprio su quello a cui stavi pensando in quel momento, e il fatto che la maggior parte delle volte che guardavi un orologio digitale trovavi quasi sempre le cifre accoppiate, tipo 13:13, 11:11 etc

Lui disse: “Sincronicità\ldots

vedi, molti si sono dedicati a studiarle, incluso Carl Jung che ha introdotto il termine, ma nessuno ha mai veramente capito il perchè e il come queste accadano.

Visto che ti stanno capitando massicciamente, ti vorrei prestare un libro al quale sono molto legato per motivi personali.”

Il libro era: \textit{La Profezia di Celestino}.

Ti sforzasti di leggerlo e lo trovasti interessante, anche se quel libro non ti chiarì cosa fossero queste coincidenze che ti capitavano a valanga.

\section{Matrimonio e tradimento}
\label{matrimonio_e_tradimento}

Arrivò il gran giorno del Matrimonio \ldots

Eri molto emozionato quel giorno, quasi cadevi a terra quando partì la marcia nunziale nella sala del comune. Capivi che non ti eri rimesso del tutto.

Fu molto dura, ritornare ad una vita normale, la depressione era incombente, il medicinale ``zombizzante'' che prendevi fu ridotto di dose, fino a un terzo.

Capisti che il tuo rapporto con tua moglie era cambiato dopo 6 anni insieme, forse anche per quello che era successo. Adesso era sempre acida con te e insultava continuamente te e la tua famiglia. Dormiva in un'altra stanza e, di lì a poco, ti tradì con una persona del nord Italia, conosciuta al viaggio di nozze.

Soffristi molto per questo e decidesti di andartene di casa. Ritornasti dai tuoi.
 
\section{Harmaghedon}
\label{harmaghedon}
 
La depressione si attenuò fino a svanire e iniziasti di nuovo, timidamente, a leggere e a provare qualche interesse. Inoltre, lo stare lontano da lei ti aveva regalato una tranquillità persa da tanti anni.

Ritornasti ad una vita nomale e ti concentrasti sul lavoro. A quel tempo, i programmatori SLA-XL si vendevano, anche se solo fuori dall'Italia. Anche Maxking e Lik-sang li acquistavano e tu e Cecco, il progettista, ne eravate fieri. Venivano usati per il Sat-Hacking.

Passò un po' di tempo.

Con Cecco iniziasti a discutere di un nuovo progetto, un programmatore JTAG avanzato e autonomo, con memoria flash. Questa, almeno, era l'idea iniziale.

Nel frattempo eri risucito a contattare Colin, trovandolo online. Lui aveva una sua home page su internet e anche lui si unì alla discussione. Fu così che decideste di progettare invece un single board computer basato su processore ARM.

Il gruppo doveva scegliere un nome.

Un tuo amico e collega, Fabrizio, ti domandò una volta: “Perchè non lo chiami Harmaghedon?”. Ti sembrò un nome molto strano per un computer ma non desti molto peso alla cosa. Non ne sapevi semplicemente il significato perché la tua cultura religiosa era molto scarsa. Per te era solo il nome di un videogioco.

Il nome venne messo ai voti. Vinse il nome scelto da Cecco, \textit{Simplemachines One}, abbreviato \textit{Sim.One}.
